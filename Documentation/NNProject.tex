\documentclass[10pt,twocolumn,letterpaper]{article}

\usepackage{cvpr}
\usepackage{times}
\usepackage{epsfig}
\usepackage{graphicx}
\usepackage{amsmath}
\usepackage{amssymb}

% Include other packages here, before hyperref.

% If you comment hyperref and then uncomment it, you should delete
% egpaper.aux before re-running latex.  (Or just hit 'q' on the first latex
% run, let it finish, and you should be clear).
\usepackage[breaklinks=true,bookmarks=false]{hyperref}

\cvprfinalcopy % *** Uncomment this line for the final submission

\def\cvprPaperID{****} % *** Enter the CVPR Paper ID here
\def\httilde{\mbox{\tt\raisebox{-.5ex}{\symbol{126}}}}

% Pages are numbered in submission mode, and unnumbered in camera-ready
%\ifcvprfinal\pagestyle{empty}\fi
\setcounter{page}{1}
\begin{document}

%%%%%%%%% TITLE
\title{\LaTeX\ Brain Tumor Project Documentaiton}


\maketitle
%\thispagestyle{empty}

%%%%%%%%% Concept 
\begin{abstract}
\end{abstract}

%%%%%%%%% BODY TEXT
\section{Introduction}

This project Uses Data set of a differnet brain tumor\linebreak
pictures and train that data set for future discovery of this
matter.

%-------------------------------------------------------------------------
\subsection{Programming Language}

Python

\subsection{Mapping}
Mapping Data set was done through saving their paths in a text file called Map.txt
%-------------------------------------------------------------------------

\subsection{Mathematics}

Mathmatics functions used in the project are conv2d,maxpool2d,localresponsenormalization.\linebreak
These funcitons are a part of tflearn liberary.

%------------------------------------------------------------------------
\section{Code Details}
The Code follows a pattern in which it gets the mapped files from the Map.txt file, then it stores them into an X and Y arrays. The mapped files (which are the data set pictures) are reshaped
into 28x28 size and turend a gray scale so that the system is trained using alpha points to identify a tumor.
The code works at accuarcy of 0.97 and lost value of 0.2 with 10000 epochs.

%---------------------------------------------------
\section{References}

The whole project was uploaded on Github: https://github.com/Kamzoki/NNProject
You can also find the project on the same DVD which this document on, along with all the assets.
\end{document}