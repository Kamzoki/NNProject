\documentclass[10pt,twocolumn,letterpaper]{article}

\usepackage{cvpr}
\usepackage{times}
\usepackage{epsfig}
\usepackage{graphicx}
\usepackage{amsmath}
\usepackage{amssymb}

% Include other packages here, before hyperref.

% If you comment hyperref and then uncomment it, you should delete
% egpaper.aux before re-running latex.  (Or just hit 'q' on the first latex
% run, let it finish, and you should be clear).
\usepackage[breaklinks=true,bookmarks=false]{hyperref}

\cvprfinalcopy % *** Uncomment this line for the final submission

\def\cvprPaperID{****} % *** Enter the CVPR Paper ID here
\def\httilde{\mbox{\tt\raisebox{-.5ex}{\symbol{126}}}}

% Pages are numbered in submission mode, and unnumbered in camera-ready
%\ifcvprfinal\pagestyle{empty}\fi
\setcounter{page}{1}
\begin{document}

%%%%%%%%% TITLE
\title{\LaTeX\ Brain Tumor Type Classificaiton Project Documentaiton}


\maketitle
%\thispagestyle{empty}

%%%%%%%%% Concept 
\begin{abstract}
This project Uses a Dataset of images\linebreak
showing different types of brain tumor as a training set for the sake of classifying futrue samples.
\end{abstract}

%%%%%%%%% BODY TEXT
\section{Introduction}
The images in the dataset show exactly three types of brain tumor: Meningioma, Glioma, Pituitary tumor.\linebreak
The dataset contains assigned labels to each of these types: 0, 1, 2 respectively.

%-------------------------------------------------------------------------
\subsection{Programming Language}
Python

\subsection{Mapping}
Mapping the dataset images was carried out by saving their paths along with their assigned labels to a text file called Map.txt
%-------------------------------------------------------------------------

\subsection{Mathematics}

Mathmatics functions used in the project are conv2d,maxpool2d,localresponsenormalization.\linebreak
These funcitons are a part of tflearn liberary.

%------------------------------------------------------------------------
\section{Code Details}

\textbf {First, Building classifier:- }

 Loading data using imagepreloader imported from tflearn.datautils.

 \textbf {Preprossing:}

 the dataset is reshaped in order to be fed to the CNN classifier.

\textbf {Building a deep neural network:-}

We are building a 5-layers neural network using TFLearn. We need to specify the shape of our input data. In our case, each sample has a total of 784 features and we will process samples per batch to save memory, so our data input shape is [None, 28,28,1] ('None' stands for an unknown dimension, so we can change the total number of samples that are processed in a batch).

Then, the classifier is trained using the dataset and tested.
%---------------------------------------------------
\section{References}

The whole project was uploaded on Github: https://github.com/Kamzoki/NNProject
You can also find the project on the same DVD which this document on, along with all the assets.
\end{document}